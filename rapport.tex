\documentclass[a4 paper, 12pt]{article}
\usepackage[french]{babel}
\usepackage{amsmath, amsthm, amsfonts, amssymb, mathrsfs}
\usepackage{hyperref}
\usepackage{graphicx}
\usepackage{caption}
\usepackage{subcaption}
\usepackage[ruled,vlined, linesnumbered]{algorithm2e}
\usepackage{xcolor}
\usepackage{listings}

\definecolor{mGreen}{rgb}{0,0.6,0}
\definecolor{mGray}{rgb}{0.5,0.5,0.5}
\definecolor{mPurple}{rgb}{0.58,0,0.82}
\definecolor{backgroundColour}{rgb}{0.95,0.95,0.92}

\lstdefinestyle{CStyle}{
    backgroundcolor=\color{backgroundColour},   
    commentstyle=\color{mGreen},
    keywordstyle=\color{magenta},
    numberstyle=\tiny\color{mGray},
    stringstyle=\color{mPurple},
    basicstyle=\footnotesize,
    breakatwhitespace=false,     
    frame=single,    
    breaklines=true,                 
    captionpos=b,                    
    keepspaces=true,                 
    numbers=left,                    
    numbersep=5pt,                  
    showspaces=false,                
    showstringspaces=false,
    showtabs=false,                  
    tabsize=2,
    language=C
}
%__________________________________________

\hypersetup{
    colorlinks=true,
    linktoc=false,
    linkcolor=blue,
    citecolor=blue
}


\begin{document}
\begin{titlepage}
    \begin{figure}[!h]
        \centering
        \includegraphics[height = .2\textwidth]{img/logoartem.png}
    \end{figure}
    \vspace{3cm}

    \begin{center}
%        \huge{Strongly connected components algorithms : parallelization and proofs}
        \huge{Architecture des ordinateurs}
    \end{center}
    \vspace{1cm}
    \begin{center}
        \large{Département Informatique}
    \end{center}
    \vspace{1 cm}
    \begin{center}
        Erwan LEBAILLY | Vilavane LY | Vincent TRÉLAT | Benjamin ZHU
    \end{center}
    \vspace{2 cm}
    \begin{center}
        \textit{\today}
    \end{center}
    \vspace{2 cm}
    \begin{center}
        ***
    \end{center}
    
\end{titlepage}
\tableofcontents
\pagebreak

\section{Exercices}
\subsection{Exercice 1}
Avec la convention $0 \leftrightarrow \texttt{faux}$ et $1 \leftrightarrow \texttt{vrai}$, $0 \wedge 1 = \texttt{faux}$.

\subsection{Exercice 2}
On donne la table de $c_0$ :
\begin{center}
    $c_0 \colon$
    \begin{tabular}{||c | c | c ||} 
     \hline
     $a_0 \backslash b_0$ & 0 & 1 \\ [0.5ex] 
     \hline\hline
     0 & 0 & 1  \\ 
     \hline
     1 & 1 & 0 \\
     \hline
    \end{tabular}
\end{center}
On peut interpréter cette table comme la table de vérité du "ou exclusif", le \textit{xor}. Ainsi, $c_0$ coincide avec $a_0 \oplus b_0 = (a_0 \vee b_0) \wedge (\neg (a_0 \wedge b_0))$.

\subsection{Exercice 3}
a. Montrer que xor est associatif et commutatif puis recopier calcul\\

b. inclure schéma

\subsection{Exercice 4}
a. On écrit le code suivant :\\
\begin{lstlisting}[style=CStyle]
int main()
    {
        printf("Sizeof int: %lu octets\n", sizeof(int));
        printf("Sizeof short: %lu octets\n", sizeof(short));
        printf("Sizeof char: %lu octets\n", sizeof(char));
        return 0;
    }
\end{lstlisting}

La sortie est la suivante :\\
\begin{lstlisting}[frame = single]
    Sizeof int: 4 octets
    Sizeof short: 2 octets
    Sizeof char: 1 octets
\end{lstlisting}

b. On écrit le code suivant :
\begin{lstlisting}[style=CStyle]
int main()
    {
        int a = pow(2, 31);
        int b = pow(2, 31);
        int c = a + b;
        printf("%d\n", c);
        return 0;
    }
\end{lstlisting}

La sortie affiche 0, ce qui correspond bien à $2^{32} \mod(2^{32})$

\subsection{Exercice 5}
On donne ci-dessous l'écriture binaire sur 4 et 8 bits de 0, 1, -1 et -2:
\begin{center}
    \begin{tabular}{c | c | c} 
    $x$ & 4 bits & 8 bits\\
        \hline
        \hline
    0 :& 0000 & 0000 0000\\ 
    1 :& 0001 & 0000 0001\\
    -1 :& 1111 & 1111 1111\\
    -2 :& 1110 & 1111 1110\\
     \hline
    \end{tabular}
\end{center}

\subsection{Exercice 6}
a. $m_1 = 0001$ et $m_{-1} = 1001$.

b. En abusant de la notation + pour des mots : $m_0 = m_1 + m_{-1} = 1010$.

c. En suivant la règle de signes, 1010 est l'encodage de -2.

\subsection{Exercice 7}

\end{document}